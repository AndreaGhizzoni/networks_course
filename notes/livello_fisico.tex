\documentclass[12pt]{article} 
\usepackage[margin=1in]{geometry}
\usepackage{enumitem} %% for custom list
%\usepackage{graphicx} %% for images
%\usepackage{multirow} %% for tables
%\usepackage{bigints}  %% for integrals

% used for tabbed spacing
%\newcommand{\itab}[1]{\hspace{0em}\rlap{#1}}
%\newcommand{\tab}[1]{\hspace{.4\textwidth}\rlap{#1}}

\begin{document}
\date{}
%\author{Andrea Ghizzoni \\
%some other info}
\title{\vspace{-11ex}} %% used for no title
 
\maketitle

\section{Livello Fisico}\label{fisico}
Ogni mezzo trasmissivo ha una certa banda di frequenza a cui vengono trasmesse delle informazioni senza una forte 
attenuazione.

Segnali che partono da 0 fino alla frequenza massima vengono chiamati in \textit{Banda Base}, mentre i segnali che 
vengono traslati per occupare frequenze pi\'u alte vengono chiamate in \textit{Banda Passante}.

\paragraph{Teorema di Shannon} \'E possibile identificare un limite superiore della capacit\'a di un mezzo trasmissivo
sotto le ipotesi che esso attenui e aggiunga solo rumore al segnale:
\begin{center}
    $ C = B*\log_2(1+S/N) $
\end{center}
dove:
\begin{itemize}[noitemsep]
    \item C [bit/s]: capacit\'a del canale
    \item B [Hz]: banda del canale
    \item S [J]: energia del segnale per ciascun bit trasmesso
    \item N [J]: energia del rumore del canale aggiunto
\end{itemize}
Il rapporto $S/N$ \'e un numero puro e viene chiamato \textit{rapporto segnale-rumore} ed espresso in \textit{Decibel}:
\begin{center}
    $ dB = 10 * log_{10}(S/N) $
\end{center}
\clearpage


\subsection{Modulazione Digitale e Multiplexing}
Ogni mezzo trasmissivo, cablato o wireless, trasmette un segnale analogico che varia nel tempo. Per questo motivo si
usano tecniche per associare ad intervalli continui del segnale analogico, un singolo valore del segnale digitale.

Il secondo problema \'e quello della condivisione del mezzo trasmissivo, ovvero come permettere a pi\'u stazioni di 
trasmettere simultaneamente utilizzando lo stesso canale. Per risolvere questo problema si utilizzano le tecniche di 
\textit{Modulazione}, cio\'e tecniche finalizzate ad imprimere un segnale elettrico, detto  \textit{modulante} su di un 
altro segnale elettrico detto \textit{Portante}.

\subsubsection{Trasmissione in Banda Base}
Questo tipo di trasmissione prevede l'uso di tensione positiva per rappresentare un "1" (digitale) e negativa per lo 
"0". Questo approccio viene chiamato \textit{NRZ} (\textit{Non Zero Return}) il che significa che il segnale segue
l'andamento dei dati.

Presenta per\'o molti problemi tecnici, che vengono parzialmente risolti utilizzando alti tipo di codifiche (ad 
esempio la codifica di Manchester).

\paragraph{Multiplaxing a Divisione di Tempo} La tecnica pi\'u semplice per permettere a pi\'u segnali di trasmettere
su unico mezzo trasmissivo \'e quella di assegnare l'intero canale ad una singola stazione per un certo lasso di 
tempo; questa multiplazione del canale viene chiamata \textit{TDM} (\textit{Time Division Multiplexing}).

\paragraph{Efficienza di Banda} Un segnale potrebbe oscillare da 0 a 1 (ovvero da negativo a positivo) ogni 1 bit, il 
che vuol dire che la banda richiesta \'e di $B/2$ Hz se il tasso di invio dei bit (bit rate) \'e di B bit/s. Di
conseguenza non si pu\'o aumentare la velocit\'a di trasmissione senza incrementare la banda.

Una possibile soluzione sarebbe di usare quattro livelli di tensione, al posto di due, per rappresentare il segnale
digitale. Questo porterebbe ad un dimezzamento del bit rate e di conseguenza di banda, ma con l'unico svantaggio che 
richiederebbe apparecchiature pi\'u costose in quanto devono amplificare di pi\'u il segnale ed essere pi\'u \\
sensibili alle variazioni dei quattro livelli di tensione.

\paragraph{Clock Recovery} Per effettuare una corretta codifica, il ricevitore deve sapere quando un simbolo termina 
e quando inizia il successivo. Questo pu\'o essere un problema nel caso di lunghe sequenza di 1 o di 0. Una delle 
tecniche per risolvere questo genere di problematiche \'e quello di temporizzare il ricevitore.


\subsubsection{Trasmissione in Banda Passante}
Usare un segnale che non inizia ad una gamma di frequenze a zero, viene chiamato \textit{trasmissione in banda 
passante}. Un segnale in banda base pu\'o essere traslato in banda passante senza perdita di informazioni.

\paragraph{Multiplexing a Divisione di Frequenza} Per permettere a molti segnali di condividere uno stesso canale 
trasmissivo, sono nate le tecniche di multiplaxing \textit{FDM} (\textit{Frequency Division Multiplaxing}), ovvero 
divide lo spettro di frequenza in bande e ogni segnale viene traslato in quella banda.


\subsection{La Rete Telefonica}
La rete telefonica viene definita come \textit{PSTN}, ovvero \textit{Public Switched Telephone Network}, ha come 
obbiettivo la trasmissione della voce umana in modo pi\'u o meno comprensibile.

\subsubsection{Struttura del Sistema Telefonico}
Da ogni telefono partono due cavi in rame che si collegano direttamente alla centrale (chiamata che Centrale Locale).
Le connessioni di ogni telefono con la propria centrale locale viene chiamato \textit{local loop} o \textit{ultimo
miglio}.

Ogni Centrale Locale ha molti collegamenti in uscita verso uno o pi\'u centri di comunicazione, chiamati
\textit{Centrali Interrurbane}, le quali a loro volta sono collegate a grandi centri di commutazione nazionali.


\subsection{Collegamenti Locali: Modem, ADSL, Fibre}
Il collegamento locale \'e chiamato ultimo miglio, ovvero il collegamento tra la centrale del gestore telefonico e 
l'utente finale. Per utilizzare questa linea venivano utilizzati Modem fonici, ovvero che utilizzavano il canale 
vocale per trasmettere dati digitali.

\paragraph{MoDem} Il termine \textit{MoDem} significa \textit{Modulatore-Demodulatore}, ovvero un dispositivo in 
grado di Modulare una serie di segnali digitali secondo un segnale portante e Demodularlo per ottenere tutti i 
segnali originali.\\\\
Al giorno d'oggi sono sostituiti con i Modem \textit{ADSL}, ma entrambi risentono delle caratteristiche negative 
dell'ultimo miglio.

\subsubsection{Modem Telefonici}
I Modem Telefonici hanno il compito di trasmetter bit su canali analogici. Dall'altra parte del cavo il segnale viene
riconvertito in bit. Dal punto di vista logico il modem si colloca tra il computer (digitale) e il sistema telefonico
(analogico).

\subsubsection{Linee DSL}
I collegamenti \textit{DSL} (\textit{Digital Subscriber Line}), hanno portato un'incredibile miglioramento delle
prestazioni in quanto non usano la frequenza di banda della voce (300 Hz - 3400 Hz) ma l'intera banda messa a 
disposizione del mezzo trasmissivo. 

Il servizio DSL \'e stato progettato per funzionare su doppini di categoria 3 e non devono interferire con fax e 
telefoni.

La tecnica utilizzata per trasmettere i segnali \'e \textit{OFDM} (\textit{Ortogonal Frequency Division 
Multiplexing}). L'ampiezza di banda viene divisa per permettere di trasmettere  i dati vocali, e per gestire il 
downstream e l'upstream dei dati.

Si parla di \textit{ADSL} (\textit{Asymmetric DSL}) quando il downstream \'e maggiore dell'upstram.


\subsection{Commutazione}
Attualmente le reti utilizzano due diverse tecniche di commutazione: la Commutazione di Circuito e quella di 
Pacchetto.

\subsubsection{Commutazione di Circuito}
Quando si effettua una chiamata telefonica classica, l'apparecchiatura di commutazione cerca di creare un percorso
fisico completo tra il telefono chiamante e il chiamato.

\subsubsection{Commutazione di Pacchetto}
I pacchetti vengono inviati non appena disponibili, al contrario della commutazione di circuito non \'e necessario
creare un collegamento fisico tra i due apparecchi; non esiste un percorso stabilito. \'E compito dei Router 
instradare i pacchetti secondo la tecnica Store-and-Forward.

La commutazione di pacchetto non spreca banda, in quando a differenza della commutazione di circuito, non viene 
pre-allocato un quantitativo di banda per ogni utente.









\end{document}