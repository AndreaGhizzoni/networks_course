\documentclass{article}

\usepackage{fancyhdr}
\usepackage{extramarks}
\usepackage{amsmath}
\usepackage{amsthm}
\usepackage{amsfonts}
\usepackage{tikz}
\usetikzlibrary{calc,arrows}
\usepackage[plain]{algorithm}
\usepackage{algpseudocode}
\usepackage{enumitem} %% for custom list
\usepackage{graphicx} %% used to split page in two columns
\usepackage{multicol}
\usetikzlibrary{automata,positioning}
\usepackage{color}

\usepackage{mathtools}
\DeclarePairedDelimiter\ceil{\lceil}{\rceil}
\DeclarePairedDelimiter\floor{\lfloor}{\rfloor}

% using \code{come code monospaced}
\def\code#1{\texttt{#1}}

%
% Basic Document Settings
\topmargin=-0.45in
\evensidemargin=0in
\oddsidemargin=0in
\textwidth=6.5in
\textheight=9.0in
\headsep=0.25in

\linespread{1.1}

\pagestyle{fancy}
\lfoot{\lastxmark}
\cfoot{\thepage}

\renewcommand\headrulewidth{0.4pt}
\renewcommand\footrulewidth{0.4pt}

\setlength\parindent{0pt}

\setcounter{secnumdepth}{0}

% change name of the problem here
\newenvironment{homeworkProblem}[1][-1]{
    \section{CONFIGURAZIONE RETE LOCALE}
}{}
\nobreak\extramarks{TCP}{}\nobreak{}


\begin{document}

\begin{homeworkProblem}
	\textbf{\\\\\\\\\\\\\\\\\\\\\\\\Considerazioni:}	% so many spaces! for pagination..
	\begin{itemize}[noitemsep]
		\item l'indirizzo IP dell'interfaccia \code{eth0} di R1 posso assumerlo assegnato 
		      dinamicamente dall'ISP
		\item in questa configurazione dei router non ha molto senso creare una sottorete comune a 
		      tutti i router, in quanto una sottorete devrebbe condividere lo stesso dominio di 
		      broadcast.
	\end{itemize}

	\textbf{\\1)} Per assegnare gli indirizzi alle LAN considero:
	\begin{itemize}[noitemsep]
		\item pool di indirizzi per le LAN pubbliche: \code{130.175.4.0/22}
		\item pool di indirizzi per le LAN private: \code{10.0.0.0/8}
	\end{itemize}
	
	
	Essendo che tutte le LAN devono avere una sottorete in \code{/24} posso dividere il pool di 
	indirizzi pubblici nel 	seguente modo:
	\begin{center}
	\begin{tikzpicture}[>=latex]

	\coordinate (A) at (0,8);
	\coordinate (B) at (0,0);
	
	\coordinate (C) at (4,8);
	\coordinate (D) at (4,0);

	\coordinate (E) at (8,8);
	\coordinate (F) at (8,0);
	
	% draw line
	\draw[thick] (A)--(B) (C)--(D) (E) -- (F);
	% label it on the top	
	\draw (A) node[above]{\Large \code{/22}};
	\draw (C) node[above]{\Large \code{/23}};
	\draw (E) node[above]{\Large \code{/24}};
	
	\coordinate (A1) at ($(A)!.01!(B)$);
	\draw (A1) node[left]{
		\begin{tabular}{r}
			%\textit{}\\
			\code{.4.0}
		\end{tabular}
	};
	\coordinate (A2) at ($(A)!.99!(B)$);
	\draw (A2) node[left]{
		\begin{tabular}{r}
			%\textit{}\\
			\code{.7.255}
		\end{tabular}
	};
	
	\coordinate (C1) at ($(C)!.01!(D)$);
	\draw (C1) node[left]{
		\begin{tabular}{r}
			%\textit{}\\
			\code{.4.0}
		\end{tabular}
	};
	\coordinate (C2) at ($(C)!.48!(D)$);
	\draw (C2) node[left]{
		\begin{tabular}{r}
			%\textit{}\\
			\code{.5.255}\\
			\code{.6.0}
		\end{tabular}
	};
	\coordinate (C3) at ($(C)!.99!(D)$);
	\draw (C3) node[left]{
		\begin{tabular}{r}
			%\textit{}\\
			\code{.7.255}
		\end{tabular}
	};
	
	\coordinate (E1) at ($(E)!.01!(F)$);
	\draw (E1) node[left]{
		\begin{tabular}{r}
			%\textit{}\\
			\code{.4.0}
		\end{tabular}
	};
	\coordinate (E2) at ($(E)!.247!(F)$);
	\draw (E2) node[left]{
		\begin{tabular}{r}
			%\textit{}\\
			\code{.4.255}\\
			\code{.5.0}
		\end{tabular}
	};
	\coordinate (E3) at ($(E)!.48!(F)$);
	\draw (E3) node[left]{
		\begin{tabular}{r}
			%\textit{}\\
			\code{.5.255}\\
			\code{.6.0}
		\end{tabular}
	};
	\coordinate (E4) at ($(E)!.74!(F)$);
	\draw (E4) node[left]{
		\begin{tabular}{r}
			%\textit{}\\
			\code{.5.255}\\
			\code{.6.0}
		\end{tabular}
	};
	\coordinate (E5) at ($(E)!.99!(F)$);
	\draw (E5) node[left]{
		\begin{tabular}{r}
			%\textit{}\\
			\code{.7.255}
		\end{tabular}
	};
	\end{tikzpicture}
	\end{center}	
	
	\pagebreak
	Mentre scelgo il seguente pool di indirizzi provati, preso dal quello scritto sopra: 
	\code{10.42.0.0/22}
	\begin{center}
	
	
	\begin{tikzpicture}[>=latex]

	\coordinate (A) at (0,8);
	\coordinate (B) at (0,0);
	
	\coordinate (C) at (4,8);
	\coordinate (D) at (4,0);

	\coordinate (E) at (8,8);
	\coordinate (F) at (8,0);
	
	% draw line
	\draw[thick] (A)--(B) (C)--(D) (E) -- (F);
	% label it on the top	
	\draw (A) node[above]{\Large \code{/22}};
	\draw (C) node[above]{\Large \code{/23}};
	\draw (E) node[above]{\Large \code{/24}};
	
	\coordinate (A1) at ($(A)!.01!(B)$);
	\draw (A1) node[left]{
		\begin{tabular}{r}
			%\textit{}\\
			\code{.0.0}
		\end{tabular}
	};
	\coordinate (A2) at ($(A)!.99!(B)$);
	\draw (A2) node[left]{
		\begin{tabular}{r}
			%\textit{}\\
			\code{.3.255}
		\end{tabular}
	};
	
	\coordinate (C1) at ($(C)!.01!(D)$);
	\draw (C1) node[left]{
		\begin{tabular}{r}
			%\textit{}\\
			\code{.0.0}
		\end{tabular}
	};
	\coordinate (C2) at ($(C)!.48!(D)$);
	\draw (C2) node[left]{
		\begin{tabular}{r}
			%\textit{}\\
			\code{.1.255}\\
			\code{.2.0}
		\end{tabular}
	};
	\coordinate (C3) at ($(C)!.99!(D)$);
	\draw (C3) node[left]{
		\begin{tabular}{r}
			%\textit{}\\
			\code{.3.255}
		\end{tabular}
	};
	
	\coordinate (E1) at ($(E)!.01!(F)$);
	\draw (E1) node[left]{
		\begin{tabular}{r}
			%\textit{}\\
			\code{.0.0}
		\end{tabular}
	};
	\coordinate (E2) at ($(E)!.247!(F)$);
	\draw (E2) node[left]{
		\begin{tabular}{r}
			%\textit{}\\
			\code{.0.255}\\
			\code{.1.0}
		\end{tabular}
	};
	\coordinate (E3) at ($(E)!.48!(F)$);
	\draw (E3) node[left]{
		\begin{tabular}{r}
			%\textit{}\\
			\code{.1.255}\\
			\code{.2.0}
		\end{tabular}
	};
	\coordinate (E4) at ($(E)!.74!(F)$);
	\draw (E4) node[left]{
		\begin{tabular}{r}
			%\textit{}\\
			\code{.2.255}\\
			\code{.3.0}
		\end{tabular}
	};
	\coordinate (E5) at ($(E)!.99!(F)$);
	\draw (E5) node[left]{
		\begin{tabular}{r}
			%\textit{}\\
			\code{.3.255}
		\end{tabular}
	};
	\end{tikzpicture}
	\end{center}	
	
	\textit{}\newline\newline % this is for pagination
	Assegno quindi gli indirizzi alle LAN:
	\begin{center}
 	\begin{tabular}{r c l c c}
 		LAN1 & : & sub1 & = & \code{130.175.4.0/24}\\
 		LAN2 & : & sub2 & = & \code{130.175.5.0/24}\\
 		LAN3 & : & sub3 & = & \code{130.175.6.0/24}\\
 		LAN4 & : & sub4 & = & \code{130.175.7.0/24}\\\\
 		
 		LANP1 & : & sub5 & = & \code{10.42.0.0/24}\\
 		LANP2 & : & sub6 & = & \code{10.42.1.0/24}\\
 		LANP3 & : & sub7 & = & \code{10.42.2.0/24}\\
 		LANP4 & : & sub8 & = & \code{10.42.3.0/24}\\\\
 		
 		LANR1R2 & : & sub9 & = & \code{10.43.0.0/30}\\
 		LANR2R3 & : & sub10 & = & \code{10.43.0.4/30}\\
 		LANR3R4 & : & sub11 & = & \code{10.44.0.8/30}\\
 	\end{tabular}
 	\end{center}
 	
 	\pagebreak
 	\textbf{2)} Per ogni interfaccia di rete di ogni router gli assegno un indirizzo:
 	\begin{center}
 	\begin{tabular}{r r c l }
 		\textbf{R1:} & \code{eth0} & = & \code{14.132.70.4/22}\\
 			         & \code{eth1} & = & \code{130.175.4.1/24}\\
 			         & \code{eth2} & = & \code{10.43.0.1/30}\\
 			         & \code{eth3} & = & \code{10.42.0.1/24}\\\\
 			         
 	    \textbf{R2:} & \code{eth0} & = & \code{10.43.0.2/30}\\
 			         & \code{eth1} & = & \code{130.175.5.1/24}\\
 			         & \code{eth2} & = & \code{10.43.0.5/30}\\
 			         & \code{eth3} & = & \code{10.42.1.1/24}\\
 			         
 	    \textbf{R3:} & \code{eth0} & = & \code{10.43.0.6/30}\\
 			         & \code{eth1} & = & \code{130.175.6.1/24}\\
 			         & \code{eth2} & = & \code{10.43.0.9/30}\\
 			         & \code{eth3} & = & \code{10.42.2.1/24}\\\\
 			         
 	    \textbf{R3:} & \code{eth0} & = & \code{10.43.0.10/30}\\
 			         & \code{eth1} & = & \code{130.175.7.1/24}\\
 			         & \code{eth2} & = & \code{10.42.3.1/24}\\\\ 	
 	\end{tabular}
 	\end{center}
 	
 	
 	
 	\pagebreak
 	\textbf{3)} Tabella di routing di \textbf{R4}\\
 	\begin{center}
 	\begin{tabular}{l l l | p{8cm}}
  		\textbf{DST} & \textbf{IFACE} & \textbf{NH} & Dividere la tabella nelle seguenti parti: \\
  		\hline
 		\code{130.175.7.0/24} & \code{eth1} & \code{direct} & \textbf{(1)} tutte le sottoreti che raggiungo
 		                                                      direttamente\\
 		\code{10.42.1.0/24}   & \code{eth2} & \code{direct} &\\\\
 		
 		\textcolor{blue}{\code{10.42.0.0/24}} & \code{eth0} & \code{10.43.0.9/30} & \textbf{(2)} tutte le 
 		                                                                            altre sottoreti che mi 
 		                                                                            serve raggiungere\\
 		\textcolor{blue}{\code{10.42.1.0/24}} & \code{eth0} & \code{10.43.0.9/30}&\\
 		\code{10.42.2.0/24} & \code{eth0} & \code{10.43.0.9/30}&\\
 		\textcolor{blue}{\code{130.175.4.0/24}} & \code{eth0} & \code{10.43.0.9/30}&\\
 		\textcolor{blue}{\code{130.175.5.0/24}} & \code{eth0} & \code{10.43.0.9/30}&\\
 		\code{130.175.6.0/24} & \code{eth0} & \code{10.43.0.9/30}&\\\\
 		
 		\code{0.0.0.0/0} & \code{eth0} & \code{10.43.0.9/30}& \textbf{(3)} la rotta di default\\ 		
 	\end{tabular}
	\end{center}
	
	\textbf{\\OSS 1)} Come si pu\'o notare la tabella di routing ha molte entrate. Si possono ridurre se si 
	sono assegnati in modo appropriato gli indirizzi IP. Come si pu\'o vedere gli indirizzi in 
	\textcolor{blue}{blu} appartengono a blocchi di indirizzo contigui e sono destinati verso la stessa 
	interfaccia di rete e verso lo stesso indirizzo IP.

	Quindi \'e possibile accoppiarli nello stesso blocco: \code{10.42.0.0/23} e \code{130.175.4.0/23}.	

	\textbf{\\OSS 2)} Come si pu\'o anche notare, ho due interfacce che collegano due sottoreti distinte, 
	\code{eth1} ed \code{eth2}, mentre per raggiungere il resto della rete ed Internet devo inoltrare tutto 
	su \code{eth0} allo stesso indirizzo IP \code{10.43.0.9/30}.
	
	Alla luce di questo si potrebbe (\emph{solo per questo router}) evitare di scrivere tutta la parte 
	\textbf{(2)}, ma direttamente la rotta di default.
	\begin{itemize}[noitemsep]
		\item vantaggi: tabella di routing minima
		\item svantaggi: meno chiarezza
	\end{itemize}
	
	\textit{}\newline % this is for pagination
	Tabella di routing di \textbf{R3}\\
	\begin{center}
 	\begin{tabular}{l l l}
  		\textbf{DST} & \textbf{IFACE} & \textbf{NH} \\
  		\hline
 		\code{130.175.6.0/24} & \code{eth1} & \code{direct} \\
 		\code{10.42.2.0/24}   & \code{eth3} & \code{direct} \\\\
 		
 		\code{130.175.7.0/24} & \code{eth2} & \code{10.43.0.10/30}\\
        \code{10.42.3.0/24}   & \code{eth2} & \code{10.43.0.10/30}\\         		
	    \textcolor{blue}{\code{130.175.4.0/24}} & \code{eth0} & \code{10.43.0.5/30}\\
 		\textcolor{blue}{\code{130.175.5.0/24}} & \code{eth0} & \code{10.43.0.5/30}\\         		
 		\textcolor{blue}{\code{10.42.0.0/24}}   & \code{eth0} & \code{10.43.0.5/30}\\
 		\textcolor{blue}{\code{10.42.1.0/24}}   & \code{eth0} & \code{10.43.0.5/30}\\\\
 		
 		\code{0.0.0.0/0} & \code{eth0} & \code{10.43.0.9/30}\\ 		
 	\end{tabular}
	\end{center}	
	\textbf{OSS 3)} Anche in questo caso vale l'accorpamento degli indirizzi in \textcolor{blue}{blu} e la 
	\textbf{OSS (2)}.
	
	\pagebreak
	Tabella di routing di \textbf{R2}\\
	\begin{center}
 	\begin{tabular}{l l l}
  		\textbf{DST} & \textbf{IFACE} & \textbf{NH} \\
  		\hline
 		\code{130.175.5.0/24} & \code{eth1} & \code{direct} \\
 		\code{10.42.1.0/24}   & \code{eth3} & \code{direct} \\\\	
 		
 		\textcolor{blue}{\code{130.175.6.0/24}} & \code{eth2} & \code{10.43.0.6/30}\\
 		\textcolor{blue}{\code{130.175.7.0/24}} & \code{eth2} & \code{10.43.0.6/30}\\
 		\textcolor{blue}{\code{10.42.2.0/24}}   & \code{eth2} & \code{10.43.0.6/30}\\
 		\textcolor{blue}{\code{10.42.3.0/24}}   & \code{eth2} & \code{10.43.0.6/30}\\
 		
 		\code{130.175.4.0/24} & \code{eth0} & \code{10.43.0.1/30}\\
        \code{10.42.0.0/24}   & \code{eth0} & \code{10.43.0.1/30}\\\\    
 		
 		\code{0.0.0.0/0} & \code{eth0} & \code{10.43.0.9/30}\\ 		
 	\end{tabular}
	\end{center}	
	\textbf{OSS 4)} Anche in questo caso vale l'accorpamento degli indirizzi in \textcolor{blue}{blu} e la 
	\textbf{OSS (2)}.
	
	\textit{}\newline % this is for pagination
	Tabella di routing di \textbf{R1}\\
	\begin{center}
 	\begin{tabular}{l l l}
  		\textbf{DST} & \textbf{IFACE} & \textbf{NH} \\
  		\hline
 		\code{130.175.4.0/24} & \code{eth1} & \code{direct} \\
 		\code{10.42.0.0/24}   & \code{eth3} & \code{direct} \\\\	
 		
 		\textcolor{blue}{\code{130.175.5.0/24}} & \code{eth2} & \code{10.43.0.2/30}\\
 		\textcolor{blue}{\code{130.175.6.0/24}} & \code{eth2} & \code{10.43.0.2/30}\\
 		\textcolor{blue}{\code{130.175.7.0/24}} & \code{eth2} & \code{10.43.0.2/30}\\
 		\textcolor{blue}{\code{10.42.1.0/24}}   & \code{eth2} & \code{10.43.0.2/30}\\
 		\textcolor{blue}{\code{10.42.2.0/24}}   & \code{eth2} & \code{10.43.0.2/30}\\
 		\textcolor{blue}{\code{10.42.3.0/24}}   & \code{eth2} & \code{10.43.0.2/30}\\
 		
 		\code{0.0.0.0/0} & \code{eth0} & \code{10.43.0.9/30}\\ 		
 	\end{tabular}
	\end{center}	
	\textbf{OSS 5)} Anche in questo caso vale l'accorpamento degli indirizzi in \textcolor{blue}{blu} e la 
	\textbf{OSS (2)}.
	
\end{homeworkProblem}

\end{document}
